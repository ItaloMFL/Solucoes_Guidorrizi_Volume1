\documentclass[article,12pt,oneside,a4paper,brazil]{abntex2}

\usepackage[alf]{abntex2cite}
\usepackage{graphicx} % Required for inserting images
\usepackage[utf8]{inputenc}
\usepackage[brazil]{babel}
\usepackage[outline]{contour} % glow around text
\usepackage{lipsum}
\usepackage{amsmath} % For align environment
\usepackage{cancel}
\usepackage{blindtext}
\usepackage{tikz, tkz-base, tkz-fct}
\usepackage{pgfplots}
\usepackage{indentfirst}
\usepackage{multirow}
\usepackage{physics}
\usepackage[T1]{fontenc}    % para suporte a caracteres acentuados

\usetikzlibrary{angles,quotes} % for pic
\contourlength{1.2pt}

% Estou colocando 2 de espaço por cada pergunta

% Personalizando as coisas do latex
\usepackage[left=2.5cm,top=2.5cm,right=2.5cm,bottom=2.5cm]{geometry}

\title{Derivada de $e^x$ e $\ln{x}$}
\author{Italo Leite}
\date{Setembro de 2024}

\begin{document}
	
	%            \maketitle
	
	
	\begin{flushleft}
		\textbf{Exercícios 7.4}
		
		\textbf{1.} Determine a equação da reta tangente ao gráfico de $f(x) = e^x$ no ponto de abscissa 0.
		
		\textbf{resposta:}
		
		%Começar a responder a questão apartir daqui!
		
		\vspace{1em}
		
		\textbf{2.} Determine a equação da reta tangente ao gráfico de $f(x) = \ln(x)$ no ponto de abscissa 1. Esboce os gráficos de f e da reta tangente.
		
		\textbf{resposta:}
		
		%Começar a responder a questão apartir daqui!
		
		\vspace{1em}
		
		\textbf{3.} Seja $f(x) = a^x$, em que $a > 0$ e $a \neq 1$ é um real dado. Mostre que $f'(x)
		= a^x \cdot \ln(a)$	
		
		\textbf{resposta:}
		
		%Começar a responder a questão apartir daqui!
		
		\vspace{1em}
		
		\textbf{4.} Calcule $f'(x)$	
			
		\begin{adjustwidth}{0.5cm}{0cm}
			$\text{a) } f(x) = 2^x$
			
			$\text{b) } f(x) = 5^x$
		
			$\text{c) } f(x) = \pi^x$
			
			$\text{d) } f(x) = e^x$
		\end{adjustwidth}	
			
		\textbf{resposta:}
		
		%Começar a responder a questão apartir daqui!
		
		\vspace{1em}
		
		\textbf{5.} Seja $g(x)=\log_{a}{x}$, em que $a > 0$ e $a \neg 1$ e é constante. Mostre que $g'(x)=\frac{1}{x \cdot \ln{a}}$.
		
		\textbf{resposta:}
		
		Antes de iniciar a manipulação da derivada, precisamos trocar a base do logaritmo e deixa-lo em logaritmo natural.
		
		\begin{equation*}
			f(x) = \log_{a}{x} = \frac{\log_{e}{x}}{\log_{e}{a}} = \frac{\ln{x}}{\ln{a}}
		\end{equation*}
		
		Aplicando o limite e tirando o $\frac{1}{\ln{a}}$ do limite, pois $a$ não depende do $h$ quando ele tende para zero. Então a equação fica da seguinte forma:
		
		\begin{equation*}
			f'(x) = \frac{1}{\ln{a}} \cdot \lim_{h \rightarrow 0} \frac{\ln{(x + h)} - \ln{(x)}}{h}
		\end{equation*}
		
		Aplicando a propriedade operatória dos logaritmos, vamos utilizar a propriedade de subtração.
		
		\begin{equation*}
			\frac{\ln{a}}{\ln{b}} \Rightarrow \ln{a} - \ln{b}
		\end{equation*}
		
		\begin{equation*}
			f'(x) = \frac{1}{\ln{a}} \cdot \lim_{h \rightarrow 0} \left[ \frac{1}{h} \ln \left( {\frac{x +h}{x}}\right) \right]
		\end{equation*}
		
		Aplicando a propriedade da potência de logaritmos
		
		\begin{equation*}
			f'(x) = \frac{1}{\ln{a}} \cdot \lim_{h \rightarrow 0} \left[ \ln \left( {\frac{x +h}{x}}\right)^{\frac{1}{h}} \right]
		\end{equation*}
		
		Deve-se fazer uma mudança de variavel $u=\frac{h}{x} \Rightarrow h =u \cdot x$. Lembrando que deve mudar também que quando $h \rightarrow 0$ e $u \rightarrow 0$. Assim
		
		\begin{equation*}
			f'(x) = \frac{1}{\ln{a}} \cdot \lim_{u \rightarrow 0} \left[ \ln \left( {\frac{x + ux}{x}}\right)^{\frac{1}{ux}} \right]
		\end{equation*}
		\begin{equation*}
			f'(x) = \frac{1}{\ln{a}} \cdot \lim_{u \rightarrow 0} \left[ \frac{1}{x} \ln \left( {1 + u}\right)^{\frac{1}{u}} \right]
		\end{equation*}
				
		Como x não depende de u, podemos remover o 1/x do limite
		
		\begin{equation*}
			f'(x) = \frac{1}{\ln{a}} \cdot \frac{1}{x} \lim_{u \rightarrow 0} \left[ \ln \left( {1 + u}\right)^{\frac{1}{u}} \right]
		\end{equation*}
				
		O segundo limite fundamental diz que:
		
		\begin{equation*}
			e = \lim_{x \rightarrow 0} \left[ \ln \left( {x + h}\right)^{\frac{1}{h}} \right]
		\end{equation*}
		
		Aplicando essa propriedade na conta vai dar um $\ln{e} = 1$.
		
		\begin{equation*}
			f'(x) = \frac{1}{x \cdot \ln{a}}
		\end{equation*}
		
		
		\vspace{1em}
		
		\textbf{6.} Calcule o $g'(x)$.
		
		\begin{adjustwidth}{0.5cm}{0cm}
			$\text{a) } g(x) = \log_{3}{x}$
			
			$\text{b) } g(x) = \log_{5}{x}$
			
			$\text{c) } g(x) = \log_{\pi}{x}$
			
			$\text{d) } g(x) = \log_{e}{x} = \ln{x}$
		\end{adjustwidth}	
		
		\textbf{resposta:}
		
		%Começar a responder a questão apartir daqui!
		
		\vspace{1em}
	\end{flushleft}
	

	
	
\end{document}